\documentclass{article}
\usepackage[utf8]{inputenc}
\usepackage{amsmath, amssymb, amsfonts, amsthm}
\usepackage{graphicx}
\usepackage{tikz}
\usepackage{wrapfig}
\usepackage{listings}
\begin{document}

\lstset
{
language=Ada,
numbers=left,
numberstyle=\tiny\color{gray},
stepnumber=1,
backgroundcolor=\color{white},
showspaces=false,
showstringspaces=false,
showtabs=false,
frame=single
}
\section{Introduction}
The GUI framework is build to provide simple GUI controls and to be compatible with any rendering interface chosen.

Only a few design choices have to be understood to use and develop GUI objects for any application.
Contexts provide the basic infrastructure by e.g. creating a window.
The GUI objects chapter explains how GUI objects are created and arranged on the screen.
Then the Canvases are introduced which provide graphical content for the GUI objects.

\section{GUI Contexts}
A context is an abstract object which must be created before any other GUI operation can be performed.
After creation it provides a three layer desktop consisting of
\begin{itemize}
 \item WindowArea
 \item ModalArea
 \item ContextArea
\end{itemize}
where GUI objects can be placed.

The WindowArea is where all ordinary objects should be placed unless you wish to use the ModalArea.
The ModalArea, once the modal mode is entered, hides the WindowArea behind a partially transparent
surface and prohibits access to any objects in the WindowArea.
Otherwise the WindowArea and the ModalArea behave the same.

The ContextArea is used only by other GUI objects to show temporary extensions like popup menues
or dropdown lists. If anything outside the active objects is clicked, a special handler is
called and all objects are removed from the ContextArea afterwards.

A small example to create a Context,
\begin{lstlisting}
with GUI;
with GUI.UseImplementations;
procedure CreateContext is
   Implementation : GUI.Implementation_Type;
   Context        : GUI.Context_Type;
begin
   GUI.UseImplementations.Register;
   Implementation:=GUI.Implementations
     .FindImplementation("xlib");
   Context:=Implementation.NewContext;

   -- Use the context

   Implementation.FreeContext(Context);
   GUI.UseImplementations.Unregister;
end CreateContext;
\end{lstlisting}

"xlib" is only one possible implementation, but configuration details are omitted here for readability.

The coordinate system in each context begins in the top left corner and counts 1 for each pixel
to the right or to the bottom.

\section{GUI objects}
The basic abstract GUI object from which all GUI objects are derived is GUI.Object\_Type.
Most GUI objects have a parent, sisters and children which means that they are inside
of another object, before or behind another object and may contain other objects which
are drawn on top of them.

All this is done by nesting rectangles. An ordinary rectangle is given for each GUI object
and called Bounds (accessed by GetBounds or SetBounds). It describes its position relative
the the parent rectangle. The bounds also include information about the visibility of the
object.

Internally the relative rectangles are nested on every change to absolute rectangles,
which include information not only on the visual part but on the necessary cuts in the
top left corner.

The above Context example can be extended showing how to place an empty object in the WindowArea.
\begin{lstlisting}
   SomeObject : GUI.Object_Access;
...
   SomeObject:=new GUI.Object_Type;
   SomeObject.Initialize(Context.WindowArea);
   SomeObject.SetBounds
     (Top     => 10,
      Left    => 10,
      Height  => 100,
      Width   => 200,
      Visible => True);
   -- Use the object
...
\end{lstlisting}
Note that this code does not produce any visible object since the plain GUI.Object\_Type creates no
canvases
There is no memory leak since Context deletes all associated objects once freed.
The same is true for child objects if an object is deleted manually.

\section{Canvases}

A canvas is a container for a picture with a number of simple drawing routines associated with it.
It can be placed within a GUI object with similar properties as a child GUI object therefore
it covers an arbitrary child rectangle.
The content of the canvas is then stretched to fill the rectangle.
Colours are in BGRA format with each component using 8 bits.

The GUI objects example is again extended to show a simple coloured blue bar,
\begin{lstlisting}
...
   SomeCanvas : GUI.Canvas_ClassAccess;
...
   SomeCanvas:=SomeObject.NewCanvas
     (Height => 1,
      Width  => 1);
   SomeCanvas.Clear(16#FF0000FF#);
   SomeCanvas.SetBounds
     (Top     => 0,
      Left    => 0,
      Height  => 100,
      Width   => 200,
      Visible => True);
...
\end{lstlisting}
Again the Canvas will be freed once the Object is destroyed but can be
freed manually using GUI.FreeCanvas.

\section{Extended GUI type}

Extended GUI type provide interfaces for GUI objects like GUI.Button.Button\_Type.
Depending on the purpose of the object some of the functionality may be implemented
with an extended GUI object. In cases of basis objects this may include visual
output, for example ListBasis\_Type which implements a list component without scrollbar
to be used as ingredient for themes (see next chapter).

\section{Themes}

A theme is a set of implementations for extended GUI types. It includes objects like
Buttons and Comboboxes.
A theme is used like an implementation, using the same infrastructure.
A theme must be registered globally and can then be found with
\begin{lstlisting}
  procedure FindTheme is
    Theme : GUI.Themes.Implementation_Type;
  begin
    Theme:=GUI.Themes.Implementations
      .FindImplementation("yellowblue");
  end FindTheme;
\end{lstlisting}
\end{document}